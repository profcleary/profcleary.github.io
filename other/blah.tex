\documentstyle[12pt]{article}
%\documentstyle{article}
\pagestyle{headings}
\textwidth=6.25in
\textheight=9in
\hoffset=-.50in
\parindent=0in
\voffset=-1.in
\begin{document}
%\input epsf
\begin{titlepage}
\newcommand{\ul}{\underline}
\newcommand{\spa}{\hspace{.15in}}
%
%
{\Large
\begin{tabular}{ll}
NAME (Printed) & \ul{\hspace{3in}}  \vspace{.45in} \\
NAME (Signed) & \ul{\hspace{3in}} \vspace{.25in} \\
\end{tabular} 
}
{\Large
\begin{center}
MATH 203 \\ EXAM 3 \\ December 4, 2001\\ Dr. Cleary
\bigskip
\medskip

\begin{tabular}{|c|l|}
\hline
\hspace{.05in} 1 \hspace{.05in} & \hspace{.5in} \mbox{ } \\ \hline
\hspace{.05in}  2 \hspace{.05in} & \hspace{.5in} \mbox{ } \\ \hline
\hspace{.05in}  3 \hspace{.05in} & \hspace{.5in} \mbox{ } \\ \hline
\hspace{.05in}  4 \hspace{.05in} & \hspace{.5in} \mbox{ } \\ \hline
\hspace{.05in}  5 \hspace{.05in} & \hspace{.5in} \mbox{ } \\ \hline
\hspace{.05in}  6 \hspace{.05in} & \hspace{.5in} \mbox{ } \\ \hline
\hspace{.05in}  7 \hspace{.05in} & \hspace{.5in} \mbox{ } \\ \hline
%\hspace{.05in}  8 \hspace{.05in} & \hspace{.5in} \mbox{ } \\ \hline
%\hspace{.05in}  9 \hspace{.05in} & \hspace{.5in} \mbox{ } \\ \hline
%\hspace{.05in}  10\hspace{.05in} & \hspace{.5in} \mbox{ } \\ \hline


\hspace{.05in} TOTAL \hspace{.05in} & \hspace{.5in} \mbox{ } \\ \hline 
\hspace{.05in} POSSIBLE \hspace{.05in} & 180  \\ \hline 
\end{tabular} 

\end{center}
}
%\parbox{3in}
%{

\bigskip

{\bf Instructions:}  Read each question carefully.
There are
a total of 7 questions on 5
pages, not counting this page.
%}


\medskip
Each question is worth 20 points, except numbers 1 and 6 which are worth 40 points.
Be sure to show your work and read the
questions carefully.  Remember that it is your obligation to answer
the question clearly and in a way that convinces the grader that
you understand how to solve the problem, so make sure your work
is clear and easy to follow.  There is no need to simplify your
answers.

\medskip
Notes, calculators and books are not to be used.
%Calculators should
%not be needed but may be used if desired.  If a calculator
%is used for a particular step in a problem, you must
%indicate that with the symbol of a circled c (``\copyright")
%at that step.
All work on this exam is to be your own.
Stop working immediately at the end of the exam
when time is called.
\end{titlepage}
%
\newtheorem{question}{Question}
\newcommand{\quest}
{\begin{question} \em \noindent
}


\newcommand{\ds}
{\displaystyle}

\newcommand{\quend}
{\end{question}}


\newcommand{\ii}
 {\mbox{$\bf \hat{\mbox{\i} }$}}

\newcommand{\jj}
 {\mbox{$\bf \hat{\mbox{\j} }$}}

\newcommand{\kk}
 {\mbox{$\bf \hat{\mbox{k} }$}}

\newcommand{\nhat}
 {\mbox{$\bf \mbox{n}$}}

\newcommand{\tang}
 {\mbox{$\bf \mbox{T}$}}

\newcommand{\del}{\partial}

\newcommand{\tfm}
 {\center {\bf \mbox{True \hspace{1in} False \hspace{1in} Meaningless}}}



\newpage


\quest Do the following series converge?  For each series,
state which test or tests were used in your analysis.
\quend

\noindent a) 
$\displaystyle \sum_{\ds n=1}^{\infty} \frac{1}{2n+3}$ 

\vspace{1.3in}
%\cdt

\noindent b) $\displaystyle \sum_{\ds n=1}^{\infty} \frac{\sqrt[\ds 4]{n}+\sqrt{n}}{n^{\ds 2}+4n} $ 
%\cdt

\vspace{1.3in}

\noindent c) $\displaystyle \sum_{\ds n=1}^{\infty} \frac{n^{\ds 2}+4n+3}{3^{\ds n+1}} $ 
%\cdt

\vspace{1.3in}

\noindent d) $\ds \frac{\ds \sqrt{2}}{\ds 3 \cdot 4} +
 \frac{\ds \sqrt{4}}{\ds 5 \cdot 6}+
\frac{\ds \sqrt{6}}{\ds 7 \cdot 8} +
\frac{\ds \sqrt{8}}{\ds 9 \cdot 10} + \cdots
 $ 
%\cdt
\vspace{1.3in}

\noindent e) $\displaystyle \sum_{\ds n=1}^{\infty} 
\frac{(n-1)! 5^{\ds n}}{(n+1)!} $
%\cdt

\vspace{3.5in}

\quest Set up the triple integral to find the mass of a solid
in the first octant bounded by $x^{\ds 2} + z^{\ds 2} = 9$ and
the planes $y=2$ and $x+y=40$ if the density of the solid is
given by $\delta(x,y,z)=4+\ds \sqrt{2x+3y+4z}$. Do not
evaluate the integral.
\quend \vspace{3in}


\quest Use cylindrical coordinates to find the
volume of the solid bounded above by the paraboloid
$z= 20 -x^{\ds 2} - y^{\ds 2} $ and below
by the cone $z^{\ds 2}=x^{\ds 2} + y^{\ds 2}$.
\quend

\newpage
\quest Let $K$ be the solid part of the ball 
 $ x^{\ds 2} + y^{\ds 2} +z^{\ds 2} \leq 25$ which lies above
the plane $z=4$.  Find the coordinates of the center of
mass of $K$.
\quend \vspace{4in}

%\quest
%Find the Taylor polynomial centered at $x=0$ of degree 6 for $f(x)= 5\sin(3x)$.
%\quend

%\newpage
%\quest Do the following sequences $\{a_n\}$ converge or diverge?  If they
%converge, compute the limit of the sequence:
%
%a) $\displaystyle a_n= \frac{n^{\ds 2}(n-n^{\ds 3})}{5n^{\ds 5}+2n}$ 
%
%\vspace{.5in}
%
%b) $\displaystyle a_n= 2 + (-1)^{\ds n} \sin(\ds \frac{1}{n})$ 
%\vspace{.5in}
%
%%c) $\displaystyle a_n= \frac{ (2n)!}{n (n)!(n-1)!}$ 
%%\vspace{.6in}
%c) $\ds a_n= \frac{\ds \sqrt{n}}{\ds \ln{n}} $ 
%\vspace{.5in}
%
%d) $\displaystyle a_n= \frac{n-444\cos(n)+ \ln(n)}{n-\sqrt{n}}$ 
%\vspace{.5in}
%\quend
\quest
Set up the integral
$\ds \int\!\int\!\int_{G}
x^{\ds 2}+y^{\ds 2}+z^{\ds 3}  \ dV $ using spherical coordinates,
for $G$ the part of the solid ball of radius 4 centered at
the origin with $y \leq 0, z \geq 0, x \leq 0 $.  Do not  evaluate the integral.
\quend

\vspace{3.5in}

%\quest
%Set up a definite iterated integral to find the surface area of
%that part of the paraboloid $z=x^{\ds 2} +y^{\ds 2}$ which
%lies between the planes $z=1$ and $z=4$.
%\quend

%\vspace{3in}

%\quest Use the second partials test to
%classify the 
%critical point $(0,0)$ for the function $f(x,y)= x^2 - y^2 -  6xy$
%as a relative min, a relative max, a saddle or none of those.
%\quend

%\newpage



%\quest 
%A bluebird travels from $(3,4)$ to $(1,2)$ staying at a constant
%distance away from a kitten located at $(3,2)$.  Give a parameterization
%of a path for the bluebird that has the propery that the initial speed
%is 12.
%\quend

\vspace{3.5in}

%\quest Find the absolute extrema for the function
%$f(x,y)=x^{\ds 2} +2x -4 x y+ 4y$ on the region R given by
%$R= \{(x,y) |0 \leq x\ 2,0 \leq y \leq 3\}$
%\quend

\newpage

\quest For each of the following series, decid whether the series diverge,
converge conditionally or converge absolutely.  Explain your reasoning-
be sure to name  each convergence test you use.
\quend

\noindent a) 
$\displaystyle \sum_{\ds n=2}^{\infty}\ \  \frac{(-1)^{\ds n}}{\ds 2n^{\ds 2}+\frac{\ds 5}{\ds n}} $ 

\vspace{1.3in}
%\cdt

\noindent b) $\displaystyle \sum_{\ds n=1}^{\infty} \frac{(-1)^{\ds n+1}}{\sqrt{n}} $ 
%\cdt

\vspace{1.3in}

\noindent c) $\displaystyle \sum_{\ds n=1}^{\infty} \frac{12^{\ds n}}{n!} $ 
%\cdt

\vspace{1.3in}

\noindent d) $\displaystyle \sum_{\ds n=1}^{\infty} (-1)^{\ds n} \sqrt{ \frac{\ds n}{\ds 2n+7}} $ 
\vspace{1.3in}


%\noindent d) $\ds 
% \frac{\ds 1}{\ds 3 \cdot 4}+
%\frac{\ds 1}{\ds 4 \cdot 5} +
%\frac{\ds 1}{\ds 5 \cdot 6} + \cdots
% $ 
%%\cdt
%\vspace{1.3in}

\noindent e) $\ds \frac{\ds -1}{\ds 4 \sqrt{2}} +
 \frac{\ds 1}{\ds 9 \sqrt{3}} -
\frac{\ds 1}{\ds 16 \cdot 2} +
\frac{\ds 1}{\ds 25 \sqrt {5}} -\cdots
 $ 
%\cdt
\vspace{1.3in}


\vspace{3.5in}
%
%\quest Evaluate:
%$\ds \int_{\ds 0}^{\ds 8}\int_{\ds \sqrt[\ds 3]{y}}^{\ds 2}
%\ \ 8e^{\ds x^{\ds 4} } \,dx\ dy $
%\quend

\newpage
%\quest Use double integration to find the volume between the 
%3 coordinate planes, the plane $4x+ y=8$ and the surface $z=6xy$.
%\quend

%\quest Use double integration to find the volume between the
%cylinder $x^2+y^2 = 9$, the xy-plane and the surface $z=10-x^2-y^2$.
%\quend

\vspace{3.5in}

 \quest 
Mayor Giuliani decides to go bungie jumping off of the Brooklyn Bridge.
The first drop he travels 60 feet downwards before the
bungie cords rebound him upwards.  Then he bounces back
upwards three-quarters of the way up before falling downwards
again.  On each successive bounce, he travels three-quarters of
the distance of the previous bounce.
Write down a series expressing the total distance he will
travel (including both up and down) and compute its sum.
\quend
%Sally is going to build a model of a snowflake made out of
%wood.  At the first stage, she starts with an equilateral triangle
%with area 1 square meter.  Then she attaches 3 smaller equilateral
%triangles, with
%one glued onto the middle third of each segment of
%the original big triangle, see drawing below.  Then she attaches
%even smaller triangles to the middle third of each
%exposed segment at this stage. She continues to attach more and more
%smaller and smaller triangles, each time gluing the new triangles
%to the middle third of the existing segments. She does
%this forever.

%Write down a series which corresponds to the total area of the snowflake
%and compute its sum.
%\quend

\vspace{4in}
%\quest Suppose $f(u,v) = 3 u + 2 v$, with 
%$u(x,y,z) = 2 x + 3 y^2 z $ and
%$v(x,y,z) = 4 x+2\cos(x z)$.  Furthermore, suppose
%x,y and z are the rectangular coordinates which are
%functions of the spherical coordinate ($\rho$,$\theta$,$\phi$). 
%Compute $\ds \frac {\partial f}{\partial \theta}$

%\quest Find and classify the relative extrema of
%the function\\ $z=16-8xy-2x^{\ds 4}-2y^{\ds 4}$
%
%\quend
%
\end{document}
\newpage

%\quest
%A farmer wants to build a box-shaped barn to contain 200 cubic
% meters of
%hay. The barn will have a roof, two opposite sides which are walls
% and two
%opposite sides which have big barn doors.
% He does not need to pay for the floor since he will be using
%dirt, which is essentially free on a farm.  The walls on the sides
% cost \$2 per square meter.  The end walls with
%the large doors cost \$4 per square meter and the roof costs \$8 per square
%meter.  He would like to build the barn for the
%least total cost.  Use the method of Lagrange multipliers
%to SET UP a system of equations to find the optimal shape of
%the barn.  You do not need to solve the system; you only need
%to set it up.
%
%\quend






%\quest
%Find the equation of the tangent plane to the surface
%$z= e^{\ds xy} $ at the point $(-2,-1,e^{\ds 2})$
%\quend
%\vspace{4in}


%\vspace{4.5in}
%
%
%\quest
%Use the fact that for 2 vector functions, $\vec{F}(t)$ and $\vec{G}(t)$, \\
%$\ds (\vec{F}(t) \cdot \vec{G}(t)) '  = 
%(\vec{F'}(t) \cdot \vec{G}(t)) + 
%(\vec{F}(t) \cdot \vec{G'}(t)) $
%to prove that for a curve which lies on a sphere centered at the
%origin, the velocity vector is always perpendicular to the position
%vector.
%\quend
%\vspace{4.5in}
%
%\quest 
%
%Let L be the line through (0,-1,2) and (1,0,1).  Parameterize
%L and find the point where L crosses the plane x+y+z=12.
%\quend
%%Either prove the statement below or give
%%an example to prove that it is false:
%%
%%{\center If $\vec{F}$ is a vector field defined in space,
%%and  if
%%$\del \cdot \vec{F} = 0$ and $\del \times \vec{F} = \vec{0}$ then
%%$\vec{F}$ must be the zero vector field.
%%}
%%\quend

 \quest 
%Rudy decides to go bungie jumping off of the Brooklyn Bridge.
%The first drop he travels 120 feet downwards before the
%bungie cords rebound him upwards.  Then he bounces back
%upwards two-thirds of the way up before falling downwards
%again.  On each successive bounce, he travels two-thirds of
%the distance of the previous bounce.
%Write down a series expressing the total distance he will
%travel and compute its sum.

\end{document}
